\chapter{Introduction}
\lettrine[lines=4, loversize=-0.1, lraise=0.1]{T}{he goal} is to create an iPhone application that can automatically translate phrases related to trekking and climbing.

Climbers often find themselves in foreign countries since the mountains are located where they are, and it is not always the case that the climber knows the native tongue. Using English to communicate and ask for help is not always an option even if both parties of the conversation can speak it to some degree. This is due to the fact that climbing and trekking, like many other special areas of interest, involves quite a bit of jargon that both parties would need to know not only in their native tongue but now also in English and preferably the other party's native tongue.

To solve this, a phrasebook application would be useful. With a grammar containing jargon for this specific domain as well as some more common words and phrases it could be used to bridge some of the language gap between speakers, as well as help a climber understand climbing and trekking route descriptions in foreign languages.

There are multiple approaches to translating text. This work will concentrate on grammatically aware translation using Grammatical Framework (GF), a programming language for multilingual grammar applications. This is different from some other systems, e.g. Google Translate, that instead use statistical models to create translations.

\section{Goals}
During the course of this project there are a few tasks that need to be addressed:
\begin{itemize}
\item A sufficient amount of domain specific jargon in the target languages needs to be gathered. The jargon should consist of words and phrases that the GF resource grammar library does not cover already.
\item A domain lexicon will be created to cover the jargon. The lexicon should have instances for at least Swedish, English and German.
\item An application that uses the domain lexicon will be developed. The application should be an iPhone app based on the existing GF Android application. Due to differences in the two platforms the existing code most likely cannot be reused, and instead only used as a guideline. The new application should be easy to repurpose to use other lexicons so that it can be reused or extended for other projects.
\item Translations generated by the system need to be validated to ensure that the system works as intended.
\end{itemize}

\section{Structure}
