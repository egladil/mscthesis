% Chalmers title page
\begin{titlepage}

\AddToShipoutPicture{\backgroundpic{-4}{56.7}{fig/auxiliary/frontpage}}
\mbox{}
\vfill
\addtolength{\voffset}{2cm}
\begin{flushleft}
	{\noindent {\Huge Grammatical Framework on the iPhone using a C++ PGF parser} \\[0.5cm]
	\emph{\Large Master of Science Thesis in Automation and Mechatronics} \\[.8cm]
	
	{\huge EMIL DJUPFELDT}\\[.8cm]
	
	{\Large Department of Computer Science and Engineering\\
	\textsc{Chalmers University of Technology} \\
	\textsc{University of Gothenburg} \\
	Gothenburg, Sweden 2013 \\
	} 
	}
\end{flushleft}

\end{titlepage}
\ClearShipoutPicture
% End Chalmers title page


\pagestyle{empty}
\newpage
\clearpage
%\mbox{}
The Author grants to Chalmers University of Technology and University of Gothenburg  the non-exclusive right to publish the Work electronically and in a non-commercial purpose make it accessible on the Internet. 
The Author warrants that he/she is the author to the Work, and warrants that the Work does not contain text, pictures or other material that violates copyright law.

The Author shall, when transferring the rights of the Work to a third party (for example a publisher or a company), acknowledge the third party about this agreement. If the Author has signed a copyright agreement with a third party regarding the Work, the Author warrants hereby that he/she has obtained any necessary permission from this third party to let Chalmers University of Technology and University of Gothenburg  store the Work electronically and make it accessible on the Internet.
\vfill
Grammatical Framework on the iPhone using a C++ PGF parser\\
\\
EMIL A. F. DJUPFELDT\\
\\
© EMIL A. F. DJUPFELDT, September 2013.\\
\\
Examiner: AARNE RANTA\\
\\
Chalmers University of Technology\\
University of Gothenburg\\
Department of Computer Science and Engineering\\
SE-412 96 Göteborg\\
Sweden\\
Telephone + 46 (0)31-772 1000\\
\\
\\
\\
\\
Department of Computer Science and Engineering\\
Göteborg, Sweden September 2013

\newpage
\clearpage
\thispagestyle{empty}

\begin{abstract}
This thesis introduces a domain specific grammar for Grammatical Framework, as well as an iPhone application utilising the grammar and a C++ library to make parsing of the grammar possible on systems that does not easily include support for Java or Haskell.

The grammar covers phrases and words related to mountainering. It is based on and extends the Phrasebook grammar from the Grammatical Framework.

The C++ library is a port of the existing Java parser and retains a similar API and structure, with allowances for differences in the two languages.

The iPhone application provides a graphical user interface for the C++ library and utilises the mountaineering grammar, allowing the user to easily input phrases and browse translations.
\end{abstract}

\newpage
\clearpage
\mbox{}
\newpage
\clearpage
\thispagestyle{empty}
\section*{Acknowledgements}
I would like to thank my supervisor Professor Aarne Ranta for his insightful comments during the course of this work.\\[1cm]

\hfill Emil Djupfeldt, Gothenburg 2013-09-11
\newpage
\clearpage
\mbox{}